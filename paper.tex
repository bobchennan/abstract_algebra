%% LyX 2.0.0 created this file.  For more info, see http://www.lyx.org/.
%% Do not edit unless you really know what you are doing.
\documentclass[oneside,english]{book}
\usepackage{amsthm}
\usepackage{amsmath}
\usepackage{amssymb}
\usepackage{fontspec}
\setcounter{secnumdepth}{3}
\setcounter{tocdepth}{3}
\usepackage{color}
\usepackage{float}
\usepackage{graphicx}
\usepackage{setspace}
\usepackage[unicode=true,pdfusetitle,
 bookmarks=true,bookmarksnumbered=true,bookmarksopen=true,bookmarksopenlevel=3,
 breaklinks=false,pdfborder={0 0 1},backref=false,colorlinks=true]
 {hyperref}
\hypersetup{
 unicode=false}

\makeatletter

%%%%%%%%%%%%%%%%%%%%%%%%%%%%%% LyX specific LaTeX commands.
\newcommand{\lyxmathsym}[1]{\ifmmode\begingroup\def\b@ld{bold}
  \text{\ifx\math@version\b@ld\bfseries\fi#1}\endgroup\else#1\fi}


%%%%%%%%%%%%%%%%%%%%%%%%%%%%%% Textclass specific LaTeX commands.
\numberwithin{equation}{section}
\numberwithin{figure}{section}
\theoremstyle{plain}
\newtheorem{thm}{\protect\theoremname}
  \theoremstyle{plain}
  \newtheorem{lem}[thm]{\protect\lemmaname}
  \theoremstyle{plain}
  \newtheorem{fact}[thm]{\protect\factname}

%%%%%%%%%%%%%%%%%%%%%%%%%%%%%% User specified LaTeX commands.
%中英文混排设置%
\usepackage[BoldFont,SlantFont,fallback,CJKchecksingle]{xeCJK}
\setmainfont{DejaVu Serif}%西文衬线字体
\setsansfont{DejaVu Sans}%西文无衬线字体
\setmonofont{DejaVu Sans Mono}%西文等宽字体
\setCJKmainfont{WenQuanYi Micro Hei}%中文衬线字体
\setCJKsansfont{WenQuanYi Micro Hei}%中文无衬线字体
\setCJKmonofont{WenQuanYi Micro Hei Mono}%中文等宽字体
\punctstyle{banjiao}%半角字符
 
%其他中文设置%
\XeTeXlinebreaklocale “zh”%中文断行
\XeTeXlinebreakskip = 0pt plus 1pt minus 0.1pt%左右弹性间距
\usepackage{indentfirst}%段落首行缩进
\setlength{\parindent}{2em}%缩进两个字符

 

\makeatother

\usepackage{xunicode}
\usepackage{polyglossia}
\setdefaultlanguage{english}
  \providecommand{\factname}{Fact}
  \providecommand{\lemmaname}{Lemma}
\providecommand{\theoremname}{Theorem}

\begin{document}
\begin{onehalfspace}
\renewcommand\arraystretch{1.2}%1.2表示表格中行间距的缩放比例因子(缺省的标准值为1),中文需要更多的间距 \renewcommand{\contentsname}{目录}
\renewcommand{\listfigurename}{插图目录}
\renewcommand{\listtablename}{表格目录}
\renewcommand{\refname}{参考文献}
\renewcommand{\abstractname}{摘要}
\renewcommand{\indexname}{索引}
\renewcommand{\tablename}{表}
\renewcommand{\figurename}{图}
\renewcommand\appendixname{附录}
\renewcommand{\bibname}{参考文献}
\end{onehalfspace}

\begin{onehalfspace}

\title{关于两个问题的一些研究}
\end{onehalfspace}

\begin{onehalfspace}

\author{2011级ACM班 陈楠昕 5110309028}
\end{onehalfspace}

\maketitle
\begin{onehalfspace}
摘要:本文主要研究了对称群及置换群中的正规子群问题以及Todd Coxeter算法实现问题

关键词:\textbf{60阶单群,Sn,对称群,An,置换群,Sn正规子群,An正规子群,Todd-Coxeter,Schreier
graph,HLT plus lookahead,Felsch method  ,Light,统筹,结合律测试,有限群生成}
\end{onehalfspace}

\begin{onehalfspace}
\tableofcontents{}
\end{onehalfspace}

\begin{onehalfspace}

\part{理论研究}
\end{onehalfspace}

\begin{onehalfspace}

\chapter{对称群Sn以及置换群An的一些性质}
\end{onehalfspace}
\begin{lem}
\begin{onehalfspace}
\label{Lemma1}任意两个正规子群交必为原群正规子群。\end{onehalfspace}
\end{lem}
\begin{proof}
\begin{onehalfspace}
设群$H$、$M$皆为群$G$正规子群,即$H\lyxmathsym{,}M\trianglelefteq G$,根据定义有
\begin{align*}
\forall g & \in G\text{,}\forall h\in H\text{,}ghg^{-1}\in H\\
\forall g & \in G\text{,}\forall m\in M\text{,}gmg^{-1}\in M
\end{align*}


则考虑群$H\cap M$,显然$\forall a\in H\cap M\text{,}$有
\[
\forall g\in G\text{,}gag^{-1}\in H\text{,}gag^{-1}\in M\Rightarrow gag^{-1}\in H\cap M
\]
\end{onehalfspace}
\end{proof}
\begin{lem}
\begin{onehalfspace}
\label{Lemma2}一个正规子群必为包含它的子群的正规子群。\end{onehalfspace}
\end{lem}
\begin{proof}
\begin{onehalfspace}
设群$H$为群$G$正规子群,即$H\trianglelefteq G$,根据定义有
\[
\forall g\in G\lyxmathsym{,}\forall h\in H\text{,}ghg^{-1}\in H
\]


设$H\leq M\leq G$,则显然有
\[
\forall m\in M\leq G\text{,}\forall h\in H,mhm^{-1}\in H
\]
\end{onehalfspace}
\end{proof}
\begin{thm}
\begin{onehalfspace}
\label{Theorem3}对于置换群$A_{n}$,其正规子群为$\begin{cases}
\{1\} & n=1,2\\
\{1\},A_{n} & n=3\\
\{1\} & n=4\\
\{1\},A_{n} & n\geq5
\end{cases}$\end{onehalfspace}
\end{thm}
\begin{proof}
\begin{onehalfspace}
对于$A_{1},A_{2},A_{3}$,易见结论显然成立。

考虑群$A_{4}$,非平凡子群共有8个:\end{onehalfspace}

\begin{itemize}
\begin{onehalfspace}
\item $\{(1),(12)(34)\}$
\item $\{(1),(13)(24)\}$
\item $\{(1),(14)(23)\}$
\item $\{(1),(123),(132)\}$
\item $\{(1),(124),(142)\}$
\item $\{(1),(134),(143)\}$
\item $\{(1),(234),(243)\}$
\item $B_{4}\triangleq\{(1),(12)(23),(13)(24),(14)(23)\}$\end{onehalfspace}

\end{itemize}
\begin{onehalfspace}
我们一一验证即发现只有$B_{4}$是$A_{4}$的正规子群。

对于$n\geq5$的情况,$A_{n}$皆为单群,这里证明略,详情见\cite{key-2}$P78$页。\end{onehalfspace}
\end{proof}
\begin{fact}
\begin{onehalfspace}
\label{Fact4}任意两个奇置换乘积为偶置换。\end{onehalfspace}
\end{fact}
\begin{thm}
\begin{onehalfspace}
\label{Theorem5}对于对称群$S_{n}$,其正规子群为$\begin{cases}
\{1\} & n=1,2\\
\{1\},A_{n},S_{n} & n=3\\
\{1\},B_{4},A_{4},S_{4} & n=4\\
\{1\},A_{n},S_{n} & n\geq5
\end{cases}$\end{onehalfspace}
\end{thm}
\begin{proof}
\begin{onehalfspace}
对于群$S_{1},S_{2}$,结论显然成立。

下面往证对于$n\geq3,n\neq4$时结论成立,即$S_{n}$只有$\{1\},A_{n},S_{n}$这三个正规子群。设
\[
H\trianglelefteq S_{n},H\neq\{1\},H\neq S_{n}
\]


由引理\ref{Lemma1}知,$H\cap A_{n}\trianglelefteq S_{n}$。

由引理\ref{Lemma2}知,$\because A_{n}\trianglelefteq S_{n}\therefore H\cap A_{n}\trianglelefteq A_{n}$

由定理\ref{Theorem3}知,在这种情况下,$H\cap A_{n}=\{1\}orA_{n}$。

若$H\cap A_{n}=\{1\}$,则考虑$H\setminus\{1\}$,必全部为奇置换,且应用事实\ref{Fact4}可知
\[
\forall x,y\in H\setminus\{1\},x\times y=1
\]


考虑$x=y$的情况,显然有
\[
x\times x=x\times y=1,\forall y\in H\setminus\{1\}
\]


所以$H\setminus\{1\}$中任意两个奇置换均相等。那么我们不妨假设他们都等于$z$,且
\[
z=(ij)\cdots\cdots
\]


然后设$\sigma=(jk)$,其中$k\neq j$,根据秘密武器有
\[
\sigma z\sigma^{-1}=(ik)\cdots\cdots
\]


因为是正规子群,所以有
\[
\sigma z\sigma^{-1}\in H\Rightarrow\sigma z\sigma^{-1}=(ij)\cdots\cdots
\]


矛盾,所以必然有$H\cap A_{n}=A_{n}$。

而对于$n=4$的情况,我们只须找出所有共轭类,取若干共轭类之并,逐个验证即可。\end{onehalfspace}

\end{proof}
\begin{onehalfspace}

\chapter{60阶单群性质}
\end{onehalfspace}

\begin{onehalfspace}

\section{唯一性证明}
\end{onehalfspace}
\begin{fact}
\begin{onehalfspace}
\label{Fact6}若\textup{$H=\left\langle x\right\rangle \text{且}H\leq G$},其中$|G|<\infty$,若$N_{G}(H)=\left\langle y\right\rangle $,则\textup{$\forall g\in G,gHg^{-1}\neq H\Rightarrow N_{G}\left(gHg^{-1}\right)=\left\langle z\right\rangle $,且$|\left\langle y\right\rangle |=|\left\langle z\right\rangle |$。}\end{onehalfspace}
\end{fact}
\begin{proof}
\begin{onehalfspace}
取$z=gyg^{-1}$即可,则$\left\langle z\right\rangle =\left\{ gy^{n}g^{-1}|0\leq n<|\left\langle y\right\rangle |\right\} $,易见大小与$\left\langle y\right\rangle $相同。\end{onehalfspace}
\end{proof}
\begin{fact}
\begin{onehalfspace}
\label{Fact7}若$H=\left\langle x\right\rangle \text{且}H\leq G$,其中$|G|<\infty$,那么$\forall g\in G,gHg^{-1}\neq H\Rightarrow N_{G}\left(H\right)\cap N_{G}\left(gHg^{-1}\right)=\{1\}$。\end{onehalfspace}
\end{fact}
\begin{proof}
\begin{onehalfspace}
$gHg^{-1}=\left\langle gxg^{-1}\right\rangle $,若$\exists y\in G\text{,}y\neq1\text{,}yHy^{-1}=H\text{且}ygHg^{-1}y^{-1}=gHg^{-1}$,则
\[
yg\in N_{G}(H)
\]


而由$yHy^{-1}=H\text{,}\left(yg\right)H\left(yg\right)^{-1}=H\text{,知}\left(y^{n}g\right)H\left(y^{n}g\right)^{-1}=H\text{,即}y^{n}g\in N_{G}\left(H\right)$,从而可得$g\in N_{G}\left(H\right)$,矛盾。\end{onehalfspace}
\end{proof}
\begin{thm}
\begin{onehalfspace}
60阶单群在同构意义下只有一个。\end{onehalfspace}
\end{thm}
\begin{proof}
\begin{onehalfspace}
设$G$是一个单群,故$G$的非单位同态像一定是同构像。考虑$G$的任意真子群左(右)陪集在乘法作用下的置换群,一定与$G$同构。$\because|G|\geq4!\therefore\forall H\leq G\text{,}[G:H]\geq5$。下面给出两种证明:\end{onehalfspace}

\begin{enumerate}
\begin{onehalfspace}
\item 设$n_{2}$为$G$中$Sylow-2$子群个数,$n_{3}$为$G$中$Sylow-3$子群个数,$n_{5}$为$G$中$Sylow-5$子群个数。\\
由Sylow定理可知有如下关系式:
\[
\begin{cases}
n_{2}\mid15 & n_{2}\equiv1(MOD\,2)\\
n_{3}\mid20 & n_{3}\equiv1(MOD\,3)\\
n_{5}\mid12 & n_{5}\equiv1(MOD\,5)
\end{cases}
\]
\\
由上可推出
\begin{align*}
n_{2} & =5\, or\,15\\
n_{3} & =10\\
n_{5} & =6
\end{align*}
\\
下面证明$n_{2}\neq15$。$\forall x\in G$,$order(x)=2$,考虑其中心化子个数,即$Z_{G}(x)$。因为有$Z_{G}(x)\leq G$,\\
若$5\mid|Z_{G}(x)|$,即$Z_{G}(x)$中存在5阶元,不妨设为$y$,则$x\times y$为10阶元,且有
\[
\left(x\times y\right)\left\langle y\right\rangle \left(x\times y\right)^{-1}=\left\langle y\right\rangle 
\]
则$N_{G}(\left\langle y\right\rangle )$中至少有4个10阶元,并且由事实\ref{Fact6}及事实\ref{Fact7},知10阶元个数至少有$4\times6=24$个。而3阶元共$2\times10=20$个,5阶元共$4\times6=24$个,则G中至少有$24+20+24=68$个元素,矛盾,所以$5\nmid|Z_{G}\left(x\right)|$。\\
若$3||Z_{G}(x)|$,即$Z_{G}(x)$中存在3阶元,同理可导出矛盾,所以$3\nmid|Z_{G}\left(x\right)|$。\\
所以$|Z_{G}\left(x\right)|\mid4$。所以与$x$共轭的元素至少有15个,已知元素共$20+24+15+1=60$个,所以$G$中共有1阶元1个,2阶元15个,3阶元20个,5阶元24个。\\
\\
若$n_{2}=15$,下面说明任意两个交只有$\left\{ 1\right\} $。设$|K1|=|K2|=4$,$\exists y\neq1$,$y\in K1\, and\, y\in K2$,因为$y$必为2阶元,而4阶群为交换群,所以$K1\text{、}K2$中所有元素皆为$y$正规化子,则至少有6个,矛盾。所以15个$Sylow-2$子群至少包含$15\times3=45$个,因此$n_{2}=5$。对于这5个子群左(右)陪集在乘法作用下的置换群同构于$S_{5}$的一个子群,而$S_{5}$中的60阶子群必为正规子群(指数为2),再由定理\ref{Theorem5}知$G\cong A_{5}$。
\item 同理可证$n_{3}=10\text{,}n_{5}=6$。下面证明$G$有指数为5的子群。\\
若$n_{2}=5$,则$G$的$Sylow-2$子群正规化子指数为5;\\
若$n_{2}=15$,易见存在两个$Sylow-2$子群$P1\text{、}P2$交不为单位子群,即为2阶子群$A$。因为4阶群都是交换群,所以
\[
P1\text{、}P2\leq Z_{G}(A)<G
\]
\\
因此$Z_{G}(A)$的阶数是4的倍数且大于4,又不能等于3,因此为5。\\
综上所述存在一个指数为5的子群,按子群陪集的在乘法作用下同构于$S_{5}$的一个子群,同上有$G\cong A_{5}$。\end{onehalfspace}

\end{enumerate}
\end{proof}
\begin{onehalfspace}

\section{实例}
\end{onehalfspace}

\begin{onehalfspace}

\part{算法及其实现}
\end{onehalfspace}

\begin{onehalfspace}

\chapter{Todd-Coxeter算法}
\end{onehalfspace}

\begin{onehalfspace}
关于Todd-Coxeter算法本身,在课堂上已经有详细介绍。但是对于Todd-Coxeter算法的实现,我一直留有疑惑。经过查阅相关资料,我发现目前来看并没有特别好的方法实现,而难点主要体现在以下两个方面:
\end{onehalfspace}
\begin{itemize}
\begin{onehalfspace}
\item 对于一个读入我们无法判断什么时候终止。特别如果群G是无限群,能在有限步内终止,但是对于时间复杂度方面不好保证。
\item 处理比较大的数据时不能盲目设置变量,以防图太大不好处理。\end{onehalfspace}

\end{itemize}
\begin{onehalfspace}
目前知道的最好做法以及程序具体实现参考的是\cite{key-1}
\end{onehalfspace}

\begin{onehalfspace}

\section{Schreier Graph}
\end{onehalfspace}

\begin{onehalfspace}
首先Todd-Coxeter算法在实现中是用Schreier graph来表示的。我们用一个例子来说明:

考虑群

\[
G:=<a,b;a^{3}=b^{3}=(ab)^{2}=1>
\]


以及子群
\[
H:=<a>\leq G
\]


考虑群作用$G/H$,我们用红色键头表示作用$a$,蓝色键头表示作用$b$。在Schreier Graph中每个点每种颜色的边入度为1,出度为1,对应群作用映射。Todd-Coxeter算法目标是我们要逐渐增加点与边,使得整个图每个点每种颜色边出度入度均为1,且图中任意一条生成关系(例如$abab$)对应路径形成一个封闭的环。

首先我们将$H$定义为$1$,考虑到有$a\in H$,于是有下图:
\begin{figure}[tbph]


\begin{centering}
\emph{\includegraphics[scale=0.5]{\string"res/step 1\string".png}}
\par\end{centering}

\end{figure}


两条蓝色的边提醒我们存在两个点与$b$作用相关,但是我们目前没有给他们命名。

接着我们对于$1$考虑三种关系:
\end{onehalfspace}
\begin{itemize}
\begin{onehalfspace}
\item $1$在$a$作用下不变,所以原图$1$满足$a^{3}$封闭;
\item 为了满足$b^{3}$,我们定义$2$、$3$,分别为$2:=1\times b$,$3:=2\times b$,关系$b^{3}$告诉我们这三个点形成蓝色封闭三角形。于是我们得到下图:
\begin{figure}[tbph]
\begin{centering}
\includegraphics[scale=0.5]{\string"res/step 2\string".png}
\par\end{centering}

\end{figure}

\item 检查$1$处关系$abab$,已经出现的有:
\begin{figure}[H]


\begin{centering}
\includegraphics[scale=0.5]{\string"res/step 3\string".png}
\par\end{centering}

\end{figure}
于是可以推断$2\times a=3$,进而得到下图:
\begin{figure}[H]


\begin{centering}
\includegraphics[scale=0.5]{\string"res/step 4\string".png}
\par\end{centering}

\end{figure}
\end{onehalfspace}

\end{itemize}
\begin{onehalfspace}
重复这些步骤,最终可得到:
\begin{figure}[H]


\begin{centering}
\includegraphics[scale=0.5]{\string"res/step 5\string".png}
\par\end{centering}

\end{figure}

\end{onehalfspace}

\begin{onehalfspace}

\section{HLT \&\& Felsch}
\end{onehalfspace}

\begin{onehalfspace}
上面的例子相对来说是一个非常简单的例子,因为我们没有发现任何两个量相同。更加一般的情况下我们经常碰到设置了多个值相同的量,这时候我们需要把图中一些点合并。

在数据规模比较大的情况下为了控制图的大小,我们得采取一些方法。
\end{onehalfspace}
\begin{thebibliography}{Bibliography}
\appendix
\begin{onehalfspace}
\bibitem{key-1}《The Todd–Coxeter procedure》,from Ken Brown, Cornell
University, April 2011 

\bibitem{key-2}《应用近世代数》,from 胡冠章,王殿军,July 2006
\end{onehalfspace}

\bibitem{key-1}《有限群基础》,from 王萼芳,Sep 2002

\end{thebibliography}
\begin{onehalfspace}
\addcontentsline{toc}{part}{参考文献}\end{onehalfspace}

\end{document}
