%% LyX 2.0.0 created this file.  For more info, see http://www.lyx.org/.
%% Do not edit unless you really know what you are doing.
\documentclass[oneside,english]{book}
\usepackage{amsthm}
\usepackage{amsmath}
\usepackage{amssymb}
\usepackage{fontspec}
\setcounter{secnumdepth}{3}
\setcounter{tocdepth}{3}
\usepackage{color}
\usepackage{graphicx}
\usepackage{setspace}
\usepackage[unicode=true,pdfusetitle,
 bookmarks=true,bookmarksnumbered=true,bookmarksopen=true,bookmarksopenlevel=3,
 breaklinks=false,pdfborder={0 0 1},backref=false,colorlinks=true]
 {hyperref}
\hypersetup{
 unicode=false}

\makeatletter
%%%%%%%%%%%%%%%%%%%%%%%%%%%%%% Textclass specific LaTeX commands.
\numberwithin{equation}{section}
\numberwithin{figure}{section}

%%%%%%%%%%%%%%%%%%%%%%%%%%%%%% User specified LaTeX commands.
%中英文混排设置%
\usepackage[BoldFont,SlantFont,fallback,CJKchecksingle]{xeCJK}
\setmainfont{DejaVu Serif}%西文衬线字体
\setsansfont{DejaVu Sans}%西文无衬线字体
\setmonofont{DejaVu Sans Mono}%西文等宽字体
\setCJKmainfont{Microsoft YaHei}%中文衬线字体
\setCJKsansfont{Microsoft YaHei}%中文无衬线字体
\setCJKmonofont{WenQuanYi Micro Hei Mono}%中文等宽字体
\punctstyle{banjiao}%半角字符
 
%其他中文设置%
\XeTeXlinebreaklocale “zh”%中文断行
\XeTeXlinebreakskip = 0pt plus 1pt minus 0.1pt%左右弹性间距
\usepackage{indentfirst}%段落首行缩进
\setlength{\parindent}{2em}%缩进两个字符
 

\makeatother

\usepackage{xunicode}
\usepackage{polyglossia}
\setdefaultlanguage{english}
\begin{document}
\begin{onehalfspace}

\title{关于群论中一些算法的实现}
\end{onehalfspace}

\begin{onehalfspace}

\author{2011级ACM班 陈楠昕 5110309028}
\end{onehalfspace}

\maketitle
\begin{onehalfspace}
Todd-Coxeter,Schreier graph,HLT plus lookahead,Felsch method  ,Light,统筹,结合律测试,有限群生成
\end{onehalfspace}

\begin{onehalfspace}
\tableofcontents{}
\end{onehalfspace}

\begin{onehalfspace}

\part{理论研究}


\part{算法及其实现}
\end{onehalfspace}

\begin{onehalfspace}

\chapter{Todd-Coxeter算法}
\end{onehalfspace}

\begin{onehalfspace}
关于Todd-Coxeter算法本身,在课堂上已经有详细介绍。但是对于Todd-Coxeter算法的实现,我一直留有疑惑。经过查阅相关资料,我发现目前来看并没有特别好的方法实现,而难点主要体现在以下两个方面:
\end{onehalfspace}
\begin{itemize}
\begin{onehalfspace}
\item 对于一个读入我们无法判断什么时候终止。特别如果群G是无限群,能在有限步内终止,但是对于时间复杂度方面不好保证。
\item 处理比较大的数据时不能盲目设置变量,以防图太大不好处理。\end{onehalfspace}

\end{itemize}
\begin{onehalfspace}
目前知道的最好做法以及程序具体实现参考的是
\end{onehalfspace}


\section{Schreier Graph}

\begin{onehalfspace}
首先Todd-Coxeter算法在实现中是用Schreier graph来表示的。我们用一个例子来说明:

考虑群

\[
G:=<a,b;a^{3}=b^{3}=(ab)^{2}=1>
\]


以及子群
\[
H:=<a>\leq G
\]

\end{onehalfspace}

考虑群作用$G/H$,我们用红色键头表示作用$a$,蓝色键头表示作用$b$。在Schreier Graph中每个点每种颜色的边入度为1,出度为1,对应群作用映射。Todd-Coxeter算法目标是我们要逐渐增加点与边,使得整个图每个点每种颜色边出度入度均为1,且图中任意一条生成关系(例如$abab$)对应路径形成一个封闭的环。

首先我们将$H$定义为$1$,考虑到有$a\in H$,于是有下图:

\begin{center}
\includegraphics{\string"res/step 1\string".png}
\par\end{center}

两条蓝色的边提醒我们存在两个点与$b$作用相关,但是我们目前没有给他们命名。

接着我们对于$1$考虑三种关系:
\begin{itemize}
\item $1$在$a$作用下不变,所以原图$1$满足$a^{3}$封闭;
\item 为了满足$b^{3}$,我们定义$2$、$3$,分别为$2:=1\times b$,$3:=2\times b$,关系$b^{3}$告诉我们这三个点形成蓝色封闭三角形。于是我们得到下图:\includegraphics{\string"res/step 2\string".png}
\item 这里再此用未完成的红色键头提醒我们缺少了东西。\end{itemize}

\end{document}
